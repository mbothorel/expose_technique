% Rapport sur l'environnement de développement avec Vagrant
\documentclass[12pt,a4paper]{article}

\usepackage[francais]{babel}
\usepackage[utf8]{inputenc}
\usepackage[T1]{fontenc}
\usepackage{graphicx} % Pour intégrer des images
\usepackage{hyperref} % Liens externes
\usepackage{listings} % Code source

\usepackage{vmargin} % Pour formater la taille des marges du doc
\setpapersize{A4}
\setmargins{20mm}{15mm}{170mm}{250mm}{10mm}{0mm}{0pt}{1.2cm}

\title{Environnement de développement avec Vagrant et Docker}
\author{Maxence Bothorel, Thibaut Crouvezier}
\date{22/01/2015}

\begin{document}

\maketitle{}

\begin{center}
\includegraphics[width=10cm]{images_rapport/univ_logo.jpg}
\end{center}

\newpage{}

\tableofcontents{}

\newpage{}

\section{Introduction}
Un environnement de développement est un ensemble d'outils et de logiciels afin d'augmenter la productivité d'un développeur. Cela peut être un éditeur de texte, avec un débogeur et un compilateur. Dans notre cas, l'environnement de développement est une plateforme qui sert au développeur a essayer ses logiciels, dans un environnement cloturé afin de ne pas porter atteinte à son système en cas d'erreurs. L'environnement est de préférence portable, afin que plusieurs développeurs puissent travailler avec la même base.

\subsection{Vagrant}
\begin{center}
\includegraphics[width=12cm]{images_rapport/vagrant_logo.jpg}
\end{center}

\textbf{Vagrant} est un outil développé en Ruby, sous licence MIT pour créer des environnements de développements. Il se veut simple d'utilisation afin d'augmenter la productivité de l'utilisateur. Le développement de Vagrant à commencer en Janvier 2010 par Mitchell Hashimoto. Celui ci a travaillé sur Vagrant pendant 3 ans sur son temps libre, avant de créer l'entreprise \textbf{Hashicorp}, afin de continuer le projet à plein temps.

Vagrant créé et gère des machines virtuelles à destinations des développeurs. Ces machines permettent à l'utilisateur d'effectuer des tests dans un environnement fermé, afin de ne pas casser son système d'exploitation. 

Vagrant met en place des machines virtuelles sur VirtualBox, VMWare (Fusion et Workstation), dans le cloud grâce à AWS d'Amazon et Openstack, et également dans des conteneurs Docker et LXC. Il est disponnible sur Windows, OS X et Linux. De plus, Red Hat a créé un plugin permettant à Vagrant de fonctionner avec KVM.

Vagrant permet donc de travailler de façon sécurisée, à seul ou à plusieurs afin d'essayer des logiciels, des scripts ou autre...

\subsection{Docker}
\begin{center}
\includegraphics[width=6.5cm]{images_rapport/docker_logo.jpg}
\end{center}

\newpage

\textbf{Docker} est une plate-forme de virtualisation par conteneurs développé en Go, sous licence Apache 2.0. Le but de cette plate-forme est d'empaqueter des applications, afin de permettre le déploiement de celles-ci sur n'importe quel type d'environnement. C'est Solomon Hykes qui crée ce projet avec la contribution de deux autres développeurs, Andrea Luzzardi et François-Xavier Bourlet qui travaillaient à ses cotés au sein de \textit{dotCloud}. Le projet est officiellement distribué à partir de Mars 2013.

L'avantage de Docker est qu'il n'intègre pas de système d'exploitation, il est directement lié au système qui l'héberge, évitant ainsi d'être gourmand en ressources. Le fait qu'il soit aussi indépendant offre la possibilité d'avoir un développement totalement transparent et rapide à intégrer. 

\section{Rappels}

\subsection{Qu'est-ce que la virtualisation ?}

La virtualisation est le fait de créer une "machine virtuelle", un nouveau système d'exploitation qui s'exécutera au dessus du système principal. On dit que la machine hôte est celle ou la machine invitée est virtualisé. Cette machine virtuelle est indépendante du matérielle (excepté l'architecture du processeur).

Le système invité est entièrement indépendant de l'hôte, ce qui permet de virtualiser un Microsoft Windows 7 sur Debian 8, ou l'inverse. Ce système a ses limites. En effet, il est gourmand en ressources (processeur, RAM) puisqu'il faut disposer d'assez de puissance pour faire fonctionner plusieurs systèmes d'exploitations.





























\subsection{Qu'est-ce que la virtualisation par conteneurs ?}
\end{document}
