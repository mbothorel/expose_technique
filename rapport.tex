% Rapport sur l'environnement de développement avec Vagrant
\documentclass[12pt,a4paper]{article}

\usepackage[francais]{babel}
\usepackage[utf8]{inputenc}
\usepackage[T1]{fontenc}
\usepackage{graphicx} % Pour intégrer des images
\usepackage{hyperref} % Liens externes
\usepackage{listings} % Code source

\title{Environnements de développement avec Vagrant et Docker}
\author{Maxence Bothorel, Thibaut Crouvezier}
\date{22/01/2015}

\begin{document}

\maketitle{}

\newpage{}

\tableofcontents{}

\newpage{}

\section{Introduction}
Test de changement Git.

\subsection{Vagrant}
\textbf{Vagrant} est un outil développé en Ruby, sous licence MIT pour créer des environnements de développements. Il se veut simple d'utilisation afin d'augmenter la productivité de l'utilisateur. Le développement de Vagrant à commencer en Janvier 2010 par Mitchell Hashimoto. Celui ci a travaillé sur Vagrant pendant 3 ans sur son temps libre, avant de créer l'entreprise \textbf{Hashicorp}, afin de continuer le projet à plein temps.

Vagrant créé et gère des machines virtuelles à destinations des développeurs. Ces machines permettent à l'utilisateur d'effectuer des tests dans un environnement fermé, afin de ne pas casser son système d'exploitation. 

Vagrant met en place des machines virtuelles sur VirtualBox, VMWare (Fusion et Workstation), dans le cloud grâce à AWS d'Amazon et Openstack, et également dans des conteneurs Docker et LXC. Il est disponnible sur Windows, OS X et Linux. De plus, Red Hat a créé un plugin permettant à Vagrant de fonctionner avec KVM.

Vagrant permet donc de travailler de façon sécurisée, à seul ou à plusieurs afin d'essayer des logiciels, des scripts ou autre...

\subsection{Docker}

\section{Mise en place}

L'installation de Vagrant est simple puisqu'il est dans les dépôts Debian, 

\end{document}
