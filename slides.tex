\documentclass{beamer}

\usepackage[francais]{babel}	% Pour le français
\usepackage[utf8]{inputenc}		% Pour l'utf8
\usepackage[T1]{fontenc}		% Pour le codage latin
\usepackage{helvet}				% Police helvet
\usepackage{hyperref}			% Lien externe
\usepackage{tipa} 				% Pour le "f" de la prononciation
\usepackage{tikz}				% Pour le schéma
\usepackage{pgfplots}

\usetheme{Warsaw}				% Thème Warsaw

\title{Bucarest}
\author{Maxence Bothorel}
\institute{Licence professionnelle ASRALL,\\
	IUT Nancy Charlemagne,\\
	Nancy}
\date{14/12/2014}

\AtBeginSection[]{
		\begin{frame}
				\frametitle{Sommaire}
				\tableofcontents[currentsection, hideothersubsections]{}
		\end{frame}
}

\begin{document}

    \begin{frame}
		\maketitle{}
	\end{frame}

	\section{Introduction}
	\begin{frame}{Bucarest}
		\begin{columns}[c]
			\column{0.5\textwidth}
			\begin{figure}
				\caption{Héraldique}
				\includegraphics[scale=0.1]{images/heraldique}
			\end{figure}

			\column{0.5\textwidth}
			\begin{figure}
				\caption{Drapeau}
				\includegraphics[scale=0.1]{images/drapeau}
			\end{figure}
		\end{columns}
	\end{frame}
	
	\begin{frame}{Bucarest}
		\begin{center}
			\begin{tabular}{ll}
				Pays & Roumanie \\
				Maire & Sorin Oprescu \\
				Population & 1 883 425 hab. \\
				Densité & 8 261 hab./km\up{2} \\
			\end{tabular}
			\pause[2]{
				\begin{block}{Note}
					\textbf{Bucarest} s'écrit \textit{Bucare\c{s}ti} et se prononce /bu.ku're\textesh{}t\up{j}/
				\end{block}
			}
		\end{center}
	\end{frame}

	\section{Géographie}
	\subsection{Climat}
	\begin{frame}{Climat}
		\begin{itemize}
			\item<+->{Le climat de la Valachie est continental}
			\item<+->{Fort clivage entre les hivers rigoureux et étés moites}
			\item<+->{Précipitations au printemps et en été}
		\end{itemize}
		\begin{description}
			\item<+->[Avant]{Les arbres coupaient du vent l'hiver et maintenaient la fraicheur l'été}
			\item<+->[Maintenant]{La mode du béton et du verre accentue les contrastes climatiques}
		\end{description}
	\end{frame}

	\subsection{Relevé météorologique}
	\begin{frame}{Relevé météorologique}
		\begin{tikzpicture}
			\begin{axis}[
			xlabel={Mois},
			ylabel={Temperatures},
			xmin=1, xmax=12,
			ymin=-5, ymax=30,
			legend pos= outer north east,
			ymajorgrids=true,
			]
				\addplot[
				color=blue,
				]
				coordinates {
						(1,-4.8)(2,-4)(3,0.1)(4,4.9)(5,10.5)(6,13.6)(7,15.4)(8,14.9)(9,10.5)(10,5.4)(11,0.6)(12,-3.4)
				};
				\addlegendentry{Temp. min.}
					
				\addplot[
				color=orange,
				]
				coordinates {
						(1,-1)(2,0.8)(3,5.8)(4,11.5)(5,16.9)(6,20.7)(7,22.6)(8,22.4)(9,17.6)(10,11.7)(11,5.2)(12,0.2)
				};
				\addlegendentry{Temp. moy.}

				\addplot[
				color=red,
				]
				coordinates {
						(1,2.8)(2,5.5)(3,11.4)(4,18)(5,24)(6,27.7)(7,29.8)(8,29.8)(9,24.6)(10,17.9)(11,9.8)(12,3.7)
				};
				\addlegendentry{Temp. max.}
			\end{axis}
		\end{tikzpicture}
	\end{frame}

	\section{Transport}
	\subsection{Métro}
	\begin{frame}{Lignes en services}
		Le métro est le transport le plus fiable de la ville
		\begin{enumerate}
			\item{La ligne M1 reliant Pantelimon à Dristor}
			\item{La ligne M2 reliant Pipera à Berceni}
			\item{La ligne M3 reliant Precizei à Anghel Saligny}
			\item{La ligne M4 reliant le Parc Bazilescu à Gare du Nord}
		\end{enumerate}
	\end{frame}
	\begin{frame}{Lignes à venir}
		Il est également prévu : 
		\begin{enumerate}
			\pause[2]{
			\item{La ligne M5 (2016) reliant Pantelimon à Râul Doamnei}}
			\item{La ligne M6 (2019) reliant l'aéroport Henri-Coand\u{a} à l'aéroport Aurel-Vlaicu}
			\pause[3]{
			\item{La ligne M7 (2020) reliant Voluntari à Bragadiru}
			\item{La ligne M5 (non planifié) reliant Cranga\c{s}i à Dristor}}
		\end{enumerate}	
	\end{frame}

	\subsection{Transport de surface}
	\begin{frame}{RATB}
		La Régie Autonome des Transports de Bucarest (RATB) :
		\begin{itemize}
			\item{L'un des plus grand résaux d'Europe}
			\item{2,3 millions de passagers par jours}
			\item{Réseau d'autobus (105 lignes)}
			\item{Réseau de trolleybus\footnote{\href{https://fr.wikipedia.org/wiki/Trolleybus}{Le Trolleybus (Wikipedia)}} (18 lignes}
			\item{Réseau de tramway (25 lignes)}
			\item{Réseau de métro léger}
		\end{itemize}
	\end{frame}
\end{document}
