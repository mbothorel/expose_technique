\documentclass{beamer}

\usepackage[francais]{babel}	% Pour le français
\usepackage[utf8]{inputenc}		% Pour l'utf8
\usepackage[T1]{fontenc}		% Pour le codage latin
\usepackage{helvet}				% Police helvet
\usepackage{hyperref}			% Lien externe

\usetheme{Warsaw}				% Thème Warsaw

\title{Environnement de développement avec Vagrant et Docker}
\author{Maxence Bothorel, Thibaut Crouvezier}
\institute{Licence professionnelle ASRALL,\\
    IUT Nancy Charlemagne,\\
    Nancy}
\date{22/01/2015}

\AtBeginSection[]{
    \begin{frame}
        \frametitle{Sommaire}
        \tableofcontents[currentsection, hideothersubsections]{}
    \end{frame}
}

\begin{document}

    \begin{frame}
        \maketitle{}
    \end{frame}
    	
    \section{Introduction}
    	
    \begin{frame}
        Intro
    \end{frame}
    
    \section{Vagrant}
    \subsubsection{Introduction}
    
    \begin{frame}{Vagrant}
        \begin{center}
            \includegraphics[scale=0.1]{images_rapport/vagrant_logo.jpg}
        \end{center}
        \begin{itemize}
            \item{Développer en Ruby sous licence MIT}
            \item{Créé en 2010 par Mitchell Hashimoto}
            \pause
            \item{Utilise des machines virtuelles}
            \item{Public cible : Les développeurs}
        \end{itemize}
    \end{frame}

    \subsection{Utilisation basique}
    \begin{frame}
        \begin{itemize}
            \item{vagrant init hashicorp/trusty64}
            \item{vagrant up}
            \item{ssh}
        \end{itemize}

        \begin{itemize}
            \item{Les Vagrantfile sont les fichiers de configurations}
            \item{Vagrant va chercher le Vagrantfile en remontant l'arborescence}
        \end{itemize}
    \end{frame}

    \subsection{Fichiers de configuration}
    \begin{frame}
    \end{frame}
    
    \subsection{Partages}
    \begin{frame}
    \end{frame}

    \subsection{L'environnement ``multi-machines''}
    \begin{frame}
    \end{frame}

    \subsection{Les provisions}
    \begin{frame}
    \end{frame}

    \subsection{Créer ses propres machines}
    \begin{frame}
    \end{frame}

    \subsection{Partager ses machines virtuelles}
    \begin{frame}
    \end{frame}

% ================= Section Docker =======================
    
    \section{Docker}
    \subsection{Introduction}
    \begin{frame}
       \begin{itemize}
          \item{Développé en Go, sous licence Apache 2.0}
          \item{Créer par Salomon Hykes en Mars 2013}
          \item{Architecture légère}
          \item{}
       \end{itemize}
    \end{frame}

    \subsection{Installation}
    \begin{frame}
       \begin{itemize}
          \item{Installation rapide depuis le dépôt Debian 8}
          \item{Installation sous OS X et Windows via une machine virtuelle}
          \item{Récupération d'une image sur le dépôt Docker}
       \end{itemize}
    \end{frame}

    \subsection{Utilisation des Conteneurs}
    \begin{frame}
       \begin{itemize}
          \item{Qu'est-ce qu'un conteneur ?}
          \item{Lancer un conteneur à partir d'une image}
          \item{Déployer des outils avec le conteneur}
       \end{itemize}
       \begin{exampleblock}{}
          \em{}
       \end{exampleblock}
    \end{frame}

    \subsection{Les fichiers de configurations Dockerfile}
    \begin{frame}
       \begin{itemize}
       \end{itemize}
    \end{frame}

    \subsection{Partage et export des images Docker}
    \begin{frame}
       \begin{itemize}
       \end{itemize}
    \end{frame}

    \subsubsection{Partage par archive}
    \begin{frame}
    \end{frame}

    \subsubsection{Partage par le hub de Docker}
    \begin{frame}
    \end{frame}

% ================== Concurrence ========================

    \section{La concurrence}
    \begin{frame}
    \end{frame}

% ================== Conclusion =========================

    \section{Conclusion}
    \begin{frame}
    \end{frame}

\end{document}
