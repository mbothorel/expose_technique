\documentclass{beamer}

\usepackage[francais]{babel}	% Pour le français
\usepackage[utf8]{inputenc}		% Pour l'utf8
\usepackage[T1]{fontenc}		% Pour le codage latin
\usepackage{helvet}				% Police helvet
\usepackage{hyperref}			% Lien externe

\usetheme{Warsaw}				% Thème Warsaw

\title{Environnement de développement avec Vagrant et Docker}
\author{Maxence Bothorel, Thibaut Crouvezier}
\institute{Licence professionnelle ASRALL,\\
    IUT Nancy Charlemagne,\\
    Nancy}
\date{22/01/2015}

\AtBeginSection[]{
    \begin{frame}
        \frametitle{Sommaire}
        \tableofcontents[currentsection, hideothersubsections]{}
    \end{frame}
}

\begin{document}

    \begin{frame}
        \maketitle{}
    \end{frame}
    	
    \section{Introduction}
    	
    \begin{frame}
        Intro
    \end{frame}
    
    \section{Vagrant}
    \subsubsection{Introduction}
    
    \begin{frame}{Vagrant}
        \begin{center}
            \includegraphics[scale=0.1]{images_rapport/vagrant_logo.jpg}
        \end{center}
        \begin{itemize}
            \item{Développer en Ruby sous licence MIT}
            \item{Créé en 2010 par Mitchell Hashimoto}
            \pause
            \item{Utilise des machines virtuelles}
            \item{Public cible : Les développeurs}
        \end{itemize}
    \end{frame}

    \subsubsection{Utilisation basique}
    \begin{frame}
        \begin{itemize}
            \item{vagrant init hashicorp/trusty64}
            \item{vagrant up}
            \item{ssh}
        \end{itemize}

        \begin{itemize}
            \item{Les Vagrantfile sont les fichiers de configurations}
            \item{Vagrant va chercher le Vagrantfile en remontant l'arborescence}
        \end{itemize}
    \end{frame}

    \subsubsection{Fichiers de configuration}
    \begin{frame}
    \end{frame}
    
    \subsubsection{Partages}
    \begin{frame}
    \end{frame}

         \subsubsubsection{Partage de dossier}
         \begin{frame}
         \end{frame}

    \subsubsection{L'environnement ``multi-machines''}
    \begin{frame}
    \end{frame}

    \subsubsection{Les provisions}
    \begin{frame}
    \end{frame}

    \subsubsection{Créer ses propres machines}
    \begin{frame}
    \end{frame}

    \subsubsection{Partager ses machines virtuelles}
    \begin{frame}
    \end{frame}
    
    
    \section{Docker}
\end{document}
