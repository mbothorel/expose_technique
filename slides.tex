\documentclass{beamer}

\usepackage[francais]{babel}	% Pour le français
\usepackage[utf8]{inputenc}		% Pour l'utf8
\usepackage[T1]{fontenc}		% Pour le codage latin
\usepackage{helvet}				% Police helvet
\usepackage{hyperref}			% Lien externe
\usepackage{listings}   % code source
\usepackage{minted}

\usetheme{Warsaw}				% Thème Warsaw

\title{Environnement de développement avec Vagrant et Docker}
\author{Maxence Bothorel, Thibaut Crouvezier}
\institute{Licence professionnelle ASRALL,\\
    IUT Nancy Charlemagne,\\
    Nancy}
\date{22/01/2015}

\AtBeginSection[]{
    \begin{frame}
        \frametitle{Sommaire}
        \tableofcontents[currentsection, hideothersubsections]{}
    \end{frame}
}

\begin{document}

    \begin{frame}
        \maketitle{}
    \end{frame}
    	
    \section{Introduction}
    	
    \begin{frame}
        Intro
    \end{frame}
    
    \section{Vagrant}
    \subsection{Introduction}
    
    \begin{frame}{Vagrant}
        \begin{center}
            \includegraphics[scale=0.1]{images_rapport/vagrant_logo.jpg}
        \end{center}
        \begin{itemize}
            \item{Développer en Ruby sous licence MIT}
            \item{Créé en 2010 par Mitchell Hashimoto}
            \pause{}
            \item{Utilise des machines virtuelles}
            \item{Public cible : Les développeurs}
        \end{itemize}
    \end{frame}

    \subsection{Utilisation basique}
    \begin{frame}{Utilisation}
        \begin{itemize}
            \item{vagrant init hashicorp/trusty64}
            \item{vagrant up}
            \item{ssh}
        \end{itemize}
        \pause{}
        \begin{block}{Vagrantfile}
            Les Vagrantfile sont les fichiers de configurations. Vagrant va chercher le Vagrantfile en remontant l'arborescence.
        \end{block}
    \end{frame}

    \subsection{Configuration}
    \begin{frame}{Vagrantfile}
        Il existe plusieurs Vagrantfiles. Ils sont tous lus dans un certain ordre :
        \begin{enumerate}
            \item{Le Vagrantfile téléchargé avec la machine.}
            \item{Un Vagrantfile dans le soddier /home/user/.vagrant.d.}
            \item{Le Vagrantfile créé lors de la commande vagrant init.}
            \item{Si le dernier n'existe pas : un Vagrantfile qui concerne plusieurs machines.}
            \item{Si le dernier n'existe pas non plus : un Vagrantfile qui concerne l'hyperviseur.}
        \end{enumerate}
    \end{frame}

    \begin{frame}[containsverbatim]{Quelques exemples}
        Il existe 4 types de configurations :
        \begin{itemize}
            \item{config.vm : la machine virtuelle en général}
            \item{config.ssh : la configuration SSH de la machine hôte}
            \item{config.winrm : la connexion à une machine Windows}
            \item{config.vagrant : concerne l'hôte}
        \end{itemize}
        \begin{minted}[fontsize=\scriptsize]{ruby}
        Vagrant.configure(2) do |config|
            config.vm.network "forwarded_port", guest : 80, host : 8080
            config.vm.network "private_network", type : "dhcp"
            config.ssh.username
            config.ssh.private_key_path
            config.winrm.host
        end
        \end{minted}
    \end{frame}

    \subsection{Utilisation avancée}
    \begin{frame}{Partage de dossiers}
    \end{frame}

    \begin{frame}{Partage sur le réseau}
    \end{frame}

    \begin{frame}{Créer ses propres machines}
    \end{frame}

    \begin{frame}{Partager ses machines virtuelles}
    \end{frame}

% ================= Section Docker =======================
    
    \section{Docker}
    \subsection{Introduction}
    \begin{frame}
       \begin{itemize}
          \item{Développé en Go, sous licence Apache 2.0}
          \item{Créer par Salomon Hykes en Mars 2013}
          \item{Architecture légère}
          \item{}
       \end{itemize}
    \end{frame}

    \subsection{Installation}
    \begin{frame}
       \begin{itemize}
          \item{Installation rapide depuis le dépôt Debian 8}
          \item{Installation sous OS X et Windows via une machine virtuelle}
          \item{Récupération d'une image sur le dépôt Docker}
       \end{itemize}
    \end{frame}

    \subsection{Utilisation des Conteneurs}
    \begin{frame}
       \begin{itemize}
          \item{Qu'est-ce qu'un conteneur ?}
          \item{Lancer un conteneur à partir d'une image}
          \item{Déployer des outils avec le conteneur}
       \end{itemize}
       \begin{exampleblock}{}
          \em{}
       \end{exampleblock}
    \end{frame}

    \subsection{Les fichiers de configurations Dockerfile}
    \begin{frame}
       \begin{itemize}
       \end{itemize}
    \end{frame}

    \subsection{Partage et export des images Docker}
    \begin{frame}
       \begin{itemize}
       \end{itemize}
    \end{frame}

    \subsubsection{Partage par archive}
    \begin{frame}
    \end{frame}

    \subsubsection{Partage par le hub de Docker}
    \begin{frame}
    \end{frame}

% ================== Concurrence ========================

    \section{La concurrence}
    \begin{frame}
    \end{frame}

% ================== Conclusion =========================

    \section{Conclusion}
    \begin{frame}
    \end{frame}

\end{document}
